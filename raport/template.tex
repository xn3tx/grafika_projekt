\documentclass[10pt, a4paper]{article}

\usepackage[utf8]{inputenc}
\usepackage[T1]{fontenc}
\usepackage[polish]{babel}
\usepackage{geometry}
\geometry{
  top=2cm,
  bottom=2cm,
  left=1.5cm,
  right=1.5cm
}

\usepackage{amsmath, amssymb}
\usepackage{graphicx}
\usepackage{fancyhdr}
\usepackage{titlesec}
\usepackage{biblatex} 
\addbibresource{sample.bib} 

\usepackage{lmodern} 
\usepackage{tabularx}
\usepackage{enumitem}
\usepackage{xcolor}
\usepackage{sectsty} 
\allsectionsfont{\sffamily}

\title{
  \textbf{Dokumentacja projektu z Programowania Obiektowego i Grafiki Komputerowej 2025} \\
  \large Implementacja symulatora systemu amortyzacji samochodowej wraz ze stacją diagnostyczną wymuszającą drgania
}

\author{
  Natalia \textsc{Sampławska 197573} \\
  Martyna \textsc{Penkowska 197926}
}

\date{\today}

\pagestyle{fancy}
\fancyhf{}
\lhead{Programowanie Obiektowe i Grafika Komputerowa}
\rhead{\thepage}

\begin{document}

\maketitle

\begin{center}
  \begin{tabular}{l r}
    Okres trwania projektu: & Semestr letni roku akademickiego 2025 \\
  \end{tabular}
\end{center}

\vspace{0.1cm}

%--------------------------------------------------------------------------------------------------------------------------------------

\section{Cel}

Stworzenie symulatora systemu amortyzacji samochodowej wraz ze stacją diagnostyczną wymuszającą drgania. 

%----------------------------------------------------------------------------------------------------------------------------------------------

\section{Zaimplementowane funkcje symulatora}

\begin{tabular}{l l}
	1. & Wybór parametrów układu - masy platformy, współczynnika tłumienia i współczynnika sprężystości\\ 
	2. & Wybór sygnału wejściowego wraz z wszystkimi jego parametrami\\
	3. & Przedstawienie graficzne układu i symulacja jego ruchu przy określonych pobudzeniach\\
	4. & Wykreślanie sygnału pobudzającego i wyjściowego oraz określanie stabilności układu\\
\end{tabular}

%------------------------------------------------------------------------------------------------------------------------------------------

\section{Wykorzystane biblioteki}

%----------------------------------------------------------------------------------------------------------------------------------------

\section{Opis funkcji}

%-------------------------------------------------------------------------------------------------------------------------------------------

\section{Podsumowanie i wnioski}




\end{document}