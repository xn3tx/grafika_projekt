\documentclass[10pt, a4paper]{article}

\usepackage[utf8]{inputenc}
\usepackage[T1]{fontenc}
\usepackage[polish]{babel}
\usepackage{geometry}
\geometry{
  top=2cm,
  bottom=2cm,
  left=1.5cm,
  right=1.5cm
}

\usepackage{amsmath, amssymb}
\usepackage{graphicx}
\usepackage{fancyhdr}
\usepackage{titlesec}
\usepackage{biblatex} 
\addbibresource{sample.bib} 

\usepackage{lmodern} 
\usepackage{tabularx}
\usepackage{enumitem}
\usepackage{xcolor}
\usepackage{sectsty} 
\allsectionsfont{\sffamily}

\title{
  \textbf{Projekt z przedmiotu Programowanie Obiektowe i Grafika Komputerowa
} \\
  \large System amortyzacji samochodowej wraz ze stacją diagnostyczną wymuszającą drgania
}

\author{
  Natalia \textsc{Sampławska 197573} \\
  Martyna \textsc{Penkowska 197926}
}

\date{\today}

\pagestyle{fancy}
\fancyhf{}
\lhead{Programowanie Obiektowe i Grafika Komputerowa}
\rhead{\thepage}

\begin{document}

\maketitle

\begin{center}
  \begin{tabular}{l r}
    Okres trwania projektu: & Semestr letni roku akademickiego 2025 \\
    Prowadzący projekt: & dr inż. \textsc{Marcin Pazio}
  \end{tabular}
\end{center}

\vspace{0.1cm}

%--------------------------------------------------------------------------------------------------------------------------------------

\section{Cel}

Stworzenie symulacji systemu amortyzacji samochodowej pozwalającej na zmiany parametrów takich jak: waga, współczynniki sprężyny i tłumika oraz platformy wprawiającej układ w drgania. Dodatkowa ilustracja stabilności układu. 

%----------------------------------------------------------------------------------------------------------------------------------------------

\section{Zaimplementowane funkcje programu}


\begin{tabular}{l l}
	1. & Wybór parametrów obiektu\\ 
	2. & Wybór sygnału wejściowego\\
	3. & Generowanie charakterystyk Bodego\\
	4. & Przedstawienie graficzne sygnału wejściowego i wyjściowego\\
	5. & Wizualizacja działania układu amortyzacji\\
\end{tabular}

%------------------------------------------------------------------------------------------------------------------------------------------

\section{Opis funkcji}

\begin{enumerate}[label=\alph*.]
  \item \textbf{Wybór parametrów obiektu i sygnału wejściowego.} \par\vspace{0.1cm}
  Z poziomu interfejsu można zmienić parametry układu – współczynniki sprężystości i tłumienia, masę oraz wybrany sygnał pobudzający wraz z wartościami, które go charakteryzują 
  (amplitudą, częstotliwością, fazą, szerokością impulsu). 

  \vspace{0.1cm}

  \textcolor{blue}{\texttt{ParameterControl}} --- klasa z funkcjami odpowiedzialnymi za wizualizację wyboru parametrów oraz ich aktualizacje.
  
  \vspace{0.1cm}

  {\texttt{update\_parameters}},{\texttt{update\_simulation\_data}},{\texttt{update\_visibility}} --- funkcje aktualizujące parametry na podstawie wyboru użytkownika.
  Dodatkowe zabezpieczenia przed niepoprawnymi wartościami, które mogłyby uniemożliwić realizację zadania.

  \vspace{0.2cm}
  
  \item \textbf{Wyświetlanie okien programu.} \par\vspace{0.1cm}
  Interfejs graficzny jest zrealizowany za pomocą biblioteki Pygame. Program składa się z 2 okien, które są aktywne jednocześnie poprzez {\texttt{threading}}. Po uruchomieniu kodu otwierają się okna programu. Naciśnięcie przycisku simulate aktywuje symulacje w oknie symulacji.
  Możliwe jest również wyświetlenie charakterystyki Bodego, wykresów sygnałów wejściowych i wyjściowych po naciśnięciu odpowiadających im przycisków.

  \vspace{0.1cm}

  \textcolor{blue}{\texttt{ParameterControl}} --- klasa z wyglądem interfejsu ustawiania parametrów oraz logiką przycisków


  \vspace{0.2cm}
  
  \item \textbf{Pobudzenia wejściowe układu.} \par\vspace{0.1cm}
  Układ może być pobudzany następującymi sygnałami: 
  \begin{itemize}
  \item \textbf{Sinusoidalnym:} \( y(t) = A \sin(2\pi f t + \varphi) \)
  \item \textbf{Prostokątnym:} \( y(t) = A \cdot \operatorname{sgn}\left( \sin(2\pi f t + \varphi) \right) \)
  \item \textbf{Piłokształtnym:} \( y(t) = \left( \frac{2A}{T} \right) (t \bmod T) - A \)
  \item \textbf{Trójkątnym:} \(y(t) = A \left(1 - 4 \left| \frac{t \bmod T}{T} - \frac{1}{2} \right| \right)\)
  \item \textbf{Impulsem prostokątnym:} \( y(t) = 
  \begin{cases}
  A, & \text{dla } 0 < t < pulse width \\
  0, & \text{w przeciwnym razie}
  \end{cases}
  \)
  \item \textbf{Skokiem jednostkowym:} \( y(t) = A \cdot u(t) \)
  \item \textbf{Impulsem jednostkowym:} \(y(t) = A \cdot \delta(t)\)
\end{itemize}

\vspace{0.2cm}
  
  \item \textbf{Wyznaczenie wyjścia układu.} \par\vspace{0.1cm}
  Do wyznaczenia wyjścia wykorzystano różniczkowanie metodą Eulera. 

  \vspace{0.1cm}

  \textcolor{blue}{\texttt{InputOutputFunction}} --- klasa odpowiedzialna za obliczanie sygnału wyjściowego
  

\vspace{0.1cm}

{\texttt{euler\_output}} --- wyznaczenie kolejnych pochodnych sygnału wyjściowego za pomocą metody Eulera

{\footnotesize
\begin{align*}
y^{(4)}[k] &= \frac{a_3 \cdot u^{(3)}[k] + a_2 \cdot u^{(2)}[k] + a_1 \cdot u^{(1)}[k] + a_0 \cdot u[k] - b_3 \cdot y^{(3)}[k-1] - b_2 \cdot y^{(2)}[k-1] - b_1 \cdot y^{(1)}[k-1] - b_0 \cdot y[k-1]}{b_4} \\
y^{(3)}[k] &= y^{(3)}[k-1] + \Delta t \cdot y^{(4)}[k] \\
y^{(2)}[k] &= y^{(2)}[k-1] + \Delta t \cdot y^{(3)}[k] \\
y^{(1)}[k] &= y^{(1)}[k-1] + \Delta t \cdot y^{(2)}[k] \\
y[k] &= y[k-1] + \Delta t \cdot y^{(1)}[k]
\end{align*}
}



  \vspace{0.2cm}

  \item \textbf{Rysowanie wykresów sygnału wejściowego i wyjściowego.} \par\vspace{0.1cm}
  Do rysowania wykresów wykorzystano bibliotekę matplotlib. 
  Rysowanie wykresów realizuje funkcja {\texttt{input\_output\_plot}}
  w klasie \textcolor{blue}{\texttt{InputOutputFunction}}.

  \vspace{0.2cm}
  
  \item \textbf{Wyznaczenie charakterystyk Bodego.} \par\vspace{0.1cm}
    Za narysowanie charakterystyki amplitudowej i fazowej oraz za określenie stabilności
    odpowiedzialna jest klasa {\texttt{bode\_plot}}. Inicjalizacja klasy pobiera potrzebne dane
    oraz przygotowuje zakres kreślonego wykresu. W funkcji \textcolor{blue}{\texttt{plotting\_bode}} obliczne są charakterystyki fazowe i amplitudowe
    układu. 
  \vspace{0.2cm}

\end{enumerate}

%---------------------------------------------------------------------------------------------------------------------------

\section{Podsumowanie i wnioski}

Cel projektu został zrealizowany. Program umożliwia obliczenie oraz wizualizację zachowania układu. Użytkownik ma możliwość sterowania parametrami symulacji, wyboru pobudzenia oraz obserwacji wyników zarówno numerycznych, jak i graficznych.

\vspace{0.5cm}

\subsection*{Elementy programowania obiektowego}

W projekcie programowanie obiektowe zostało wykorzystane do podziału programu na uporządkowane klasy. Każdy istotny element układu (sprężyna, tłumik itd.) został odwzorowany za pomocą osobnych klas. Przykładowo:

\begin{itemize}
  \item \texttt{Spring} — klasa reprezentująca sprężynę, przechowuje jej model graficzny 3D,
  \item \texttt{Attenuator} — reprezentuje tłumik i jego model graficzny 3D,
  \item \texttt{Wheel} - klasa odpowiedzialna za graficzne przedstawienie koła będącego częścią układu,
  \item \texttt{InputOutputFunction} — realizuje obliczenia związane z dynamiką układu,
  \item \texttt{ParameterControl} — odpowiada za interfejs użytkownika i aktualizację parametrów,
  \item \texttt{BodePlot} — oblicza i rysuje charakterystyki Bodego,
\end{itemize}

Dzięki takiemu podziałowi kod stał się przejrzysty. Poszczególne klasy odpowiadają poszczególne zadania, co pozwala na łatwą zmianę programu bez konieczności ingerencji w cały system, ułatwia odnalezienie porządanego fragmentu kodu. Korzystano również z dziedziczenia i enkapsulacji — dane przechowywane są wewnątrz obiektów, a użytkownik korzysta z interfersu.

\vspace{0.5cm}

\subsection*{Grafika komputerowa z OpenGL}

Biblioteka OpenGL umożliwia trójwymiarową wizualizację układu dynamicznego w czasie rzeczywistym. Dzięki połączeniu Pygame (zarządzającego oknem oraz zdarzeniami) z PyOpenGL możliwe było narysowanie obiektów takich jak sprężyna, masa oraz koło. Dynamiczne zmiany w układzie odwzorowują zmiany wartości obliczanych numerycznie, co tworzy spójną reprezentację fizyki i grafiki.
Wizualizacja ta jest zsynchronizowana z częścią obliczeniową, co pozwala na intuicyjne zrozumienie wpływu parametrów na zachowanie układu.


\end{document}